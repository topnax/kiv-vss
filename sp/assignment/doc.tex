\documentclass[12pt, a4paper]{article}

\usepackage[english]{babel}
\usepackage{lmodern}
\usepackage[utf8]{inputenc}
\usepackage[T1]{fontenc}
\usepackage[pdftex]{graphicx}
\usepackage{amsmath, amssymb}
\usepackage[hidelinks,unicode]{hyperref}
\usepackage{float}
\usepackage{listings}
\usepackage{tikz}
\usepackage{xcolor}
\usepackage{tabularx}
\usepackage[final]{pdfpages}
\usepackage{syntax}
\usepackage{caption}
\usepackage{subcaption}
\usepackage{amsfonts}
\usepackage{wrapfig}


\definecolor{mauve}{rgb}{0.58,0,0.82}
\usetikzlibrary{shapes,positioning,matrix,arrows}

\newcommand{\img}[1]{(viz obr. \ref{#1})}

\definecolor{pblue}{rgb}{0.13,0.13,1}
\definecolor{pgreen}{rgb}{0,0.5,0}
\definecolor{pred}{rgb}{0.9,0,0}
\definecolor{pgrey}{rgb}{0.46,0.45,0.48}


\lstdefinestyle{flex}{
    frame=tb,
    aboveskip=3mm,
    belowskip=3mm,
    showstringspaces=false,
    columns=flexible,
    basicstyle={\small\ttfamily},
    numbers=none,
    numberstyle=\tiny\color{black},
    keywordstyle=\color{black},
    commentstyle=\color{black},
    stringstyle=\color{black},
    breaklines=true,
    breakatwhitespace=true,
    tabsize=3
}

\lstset{
    frame=tb,
    language=Python,
    aboveskip=3mm,
    belowskip=3mm,
    showstringspaces=false,
    columns=flexible,
    basicstyle={\small\ttfamily},
    numbers=none,
    numberstyle=\tiny\color{gray},
    keywordstyle=\color{blue},
    commentstyle=\color{pgreen},
    stringstyle=\color{mauve},
    breaklines=true,
    breakatwhitespace=true,
    tabsize=3
}


\let\oldsection\section
\renewcommand\section{\clearpage\oldsection}

\begin{document}
	% this has to be placed here, after document has been created
	% \counterwithout{lstlisting}{chapter}
	\renewcommand{\lstlistingname}{Ukázka kódu}
	\renewcommand{\lstlistlistingname}{Seznam ukázek kódu}
    \begin{titlepage}

        \centering

        \vspace*{\baselineskip}
        \begin{figure}[H]
        \centering
        \includegraphics[width=7cm]{img/fav-logo.jpg}
        \end{figure}

        \vspace*{1\baselineskip}

        \vspace{0.75\baselineskip}

        \vspace{0.5\baselineskip}
        {KIV/VSS Semester Project Assignment}

        {\LARGE\sc Benchmarking A Payment Terminal Managemenent Server \\}

        \vspace{4\baselineskip}

        \vspace{0.5\baselineskip}

        {\sc\Large Stanislav Král \\}
        \vspace{0.5\baselineskip}
        {A20N0091P}

        \vfill

        {\sc Západočeská univerzita v Plzni\\
        Fakulta aplikovaných věd}

    \end{titlepage}

    % TOC
    \tableofcontents
    \pagebreak

\section{Introduction}

\begin{wrapfigure}{r}{0.38\textwidth}
    \centering
    \includegraphics[width=.37\textwidth]{img/dotypay-payment-terminal.png}
    \captionof{figure}{Dotypay product landing page image}
\end{wrapfigure}

Dotypay\footnote{\url{https://dotypay.com/}} is a product that allows merchants to interface with payment cards to make electronic funds transfers via a payment terminal it offers.
The terminal is a mobile device running an Android OS with a built-in chip and contactless card reader with the support of magnetic stripe cards.
The fact that the terminal is running an Android OS means that it allows for the utilization of various cash register applications that can be installed directly on the terminal. 
This reduces the number of devices the merchant has to operate in order to process customer orders to just a single all-in-one device.

The terminal comes with two preinstalled applications: Dotypay and Dotypay Launcher. The former is used for creating and processing payment, preauthorization and return transactions with the ability to preview the transaction and settlement history. The latter is used for device configuration which consists of the following tasks:

\begin{itemize}
    \item \textbf{cryptographic keys setup} -- in order to securely process card payment transactions the terminal must be properly configured in the means of having all required cryptographic keys securely stored in the device storage,
    \item \textbf{merchant personalization} -- configuration of the terminal based on the merchant it belongs to (currency code, country code,...),
    \item and \textbf{maintenance of the software installed} -- periodically check whether there are updates available to the installed software such as payment application, terminal firmware or cash register application.

\end{itemize}

Configuration and management of terminals happens on a dedicated web application, and is also accessible to merchants to view transaction history, analytics and other useful insights of made turnovers.

Dotypay also claims to charge smaller transaction fees than other payment processing products in the market.


\section{Dotypay Portal server}

As mentioned in the introduction Dotypay payment terminals have to retrieve their configuration from a remote server (Dotypay Portal) in order to be functioning properly. 
The Dotypay Portal server is the subject to be put under a test to investigate its capability to handle terminal requests.

The main component of the server is a Java application based on the Spring Boot framework that is dependent on the following components:

\begin{itemize}
    \item \textbf{MS SQL database} -- main database used for persisting merchant data and terminal configuration,
    \item \textbf{Mongo database}  -- secondary database used to store transactions performed by terminals,
    \item \textbf{Amazon S3 storage}  -- object storage service for storing installation packages of terminal application as well as any merchant documents,
    \item \textbf{Amazon SQS} -- a queue service used for submitting transaction requests to the terminal,
    \item \textbf{SMTP server} -- a mail server for communication with merchants,
    \item and \textbf{Eddie TMS} -- a terminal management system by Monet+\footnote{Part of the EMV Payments service \url{https://www.monetplus.cz/emv-payments}}
\end{itemize}

It also depends on other components such as online web application Rejstřík\footnote{\url{www.rejstrik.cz}}, MailChimp and OTRS.

\subsection{Hardware specification}
At the time of writing this document, the Spring Boot application runs on an AWS node (2 vCPU and 4GB RAM plan), while the MS SQL database runs on another AWS node (2 vCPU and 2GB RAM plan).
The hardware configuration of other Amazon components is not known as they are supposed to be automatically scaled based on usage load as Amazon claims. The remaining components are not directly managed by Dotypay so their configuration is also not known. 

For the purposes of benchmarking a new instance of the server running on the same architecture will be made available to the student.

\subsection{Terminal synchronization with Dotypay Portal}
The Dotypay Launcher application that comes preinstalled on payment terminals is responsible for periodically checking the Dotypay Portal server for terminal configuration. It does so by creating an HTTP request to the REST API interface of the server every 15 minutes with the addition of a random delay up to 5 minutes.

During the processing of the synchronization request, the server has to query the database for the terminal configuration. After querying the database the server responds with the configuration data. 

If the configuration contains any mandatory applications or updates the terminal then proceeds to download installation packages from the server. The installation package data is fed to the terminal from the Amazon S3 storage.

In the past, there has been a case whereupon updating a version of a mandatory application in the Dotypay Portal a great number of terminals after fetching the latest configuration started to download the application installation package at once. 
This load on the server resulted in its instability and increased response time as well as failed package downloads across most of the terminals. 

\subsection{Submission of a performed transaction}
Each transaction that is performed on the terminal is submitted to the server so that the history of transactions is preserved and can be viewed at later date elsewhere than in the terminal itself.
The transaction data is also used for creating insightful reports of merchant sales.

All transactions are submitted to the server even if they haven't been finished successfully (e.g. transaction declined by a card scheme).

When the server receives a request to store an incoming transaction it performs only minor processing and forwards the transaction to a MongoDB database instance, which is currently being implemented by a MongoDB Atlas service at Amazon.

The payment terminal is currently not capable of processing transactions when the connection to the Internet is not available, which means that a batch of multiple transactions coming from a single terminal to be stored in the database is not expected. 

As the number of merchants using the Dotypay product is continuously increasing, the volume of transactions is following this trend.
Various spikes have been observed in the past in regards to the total transaction volume and count, however, no reports of service degradation have been reported so far.
For example, such spikes can be seen days before holidays, during lunchtime or after the usual time people finish their shift at work. During the rest of the day the usage of the server is somewhat evenly distributed, with the number of transactions slowly decreasing as nighttime comes by. 
In the early morning, it eventually starts to increase again.

\section{Benchmark scenarios}

An important question to ask is how many active terminals is the Dotypay Portal going to be able to manage without having to dramatically decrease the quality of the service. To attempt to answer this question three benchmark scenarios have been proposed.

\subsection{Terminal configuration synchronization}
The purpose of this scenario is to gain knowledge about the limit there may be to the number of terminals whose synchronization requests the server can handle in a short period of time. This scenario is based on the fact that terminals periodically ask the server for the latest version of terminal configuration. 

This scenario requests a certain number of terminals to be added to the database before performing the benchmark.
Requests created during the benchmark do not require to be filled with data as the server does not require any. 

\subsection{Submission of performed transactions}
This scenario simulates a submission of transactions in isolation from other tasks. Similarly to the previous scenario, this scenario attempts to find the limit may be to the number of terminals whose requests the server can handle.

Before performing the benchmark a certain amount of transaction data has to be acquired or randomly generated as each transaction submission request has to include a transaction to be stored. 

\subsection{Transaction and synchronization requests}
The third scenario is a combination of the previous two scenarios in an attempt to better model the reality. 

\section{Benchmarking web applications}
Today's developers rely on technologies that, with a minimum effort, can guarantee some decent results in terms of performance.
Concurrency and thread management is managed by modern web frameworks without requiring much effort from developers. 
However, even with these tools, it is still required to own the responsibility of writing good and optimized code.
Due to the pressure put on developers to ship software as fast as possible this is sometimes overlooked and various problems pop up when the software is published to the production.
Even a relatively small peak, if the server is not properly configured, may be able to crash the whole application and ruin its credibility.

To make web applications scale easier, various cloud service providers such as Amazon and Google have started to offer serverless technologies such as \textbf{PaaS} (Platform as a Service), \textbf{BaaS} (Backend as a Service) and \textbf{FaaS} (Function as a Service).
These technologies may help to improve the ability of web applications to behave correctly during higher loads, but nevertheless, it is still critical to have at least a basic knowledge of the properties and limits of the service we are running. 

Discovering properties and limits of a web application is done by running benchmark tests on exposed interfaces of the applications and often requires them to be run in isolation from the outer environment so that the tests are repeatable and not biased.

\subsection{Useful software}
The following software may be useful during the benchmarking of web applications:

\begin{itemize}
    \item \textbf{ab}\footnote{\url{https://httpd.apache.org/docs/2.4/programs/ab.html}}  -- a simple tool for benchmarking HTTP servers developed by Apache
   \item \textbf{Apache JMeter}\footnote{\url{http://jmeter.apache.org/}} -- a Java application designed to load functional behavior and measure performance primarily of web applications,
   \item \textbf{k6}\footnote{\url{https://k6.io/}} -- an alternative to the Apache JMeter that allows for writing load tests of REST APIs in the Javascript language,
   \item and \textbf{Python programming language} -- some simple benchmark scenarios can be implemented by writing a group of Python scripts as the Python language offers expressive grammar.
\end{itemize}

Additional software may be used to generate sample data to be used before and during benchmarking.

Also, if the mentioned software is insufficient in regards to achieving useful results other benchmarking tools may be looked for.

\section{Summary}
The proposed benchmark scenarios will be implemented and run in order to make an attempt to discover the limit that there may be to the number of devices the server can handle. 
These results may eventually be used by the development team at Dotypay to eliminate any critical bottlenecks or performance issues that may be discovered.

It is expected that the available hardware for running benchmark scenarios may not fully be able to put enough stress on the server and thus only a subtle increase in server response time will be observed.

Due to the fact that the scenarios do not cover any CPU demanding tasks (the server itself does not do any) it should be the database that becomes the main bottleneck. However, because the MongoDB database will be provided by the Amazon platform it may not be that case.





\end{document}

